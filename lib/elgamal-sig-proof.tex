%%% Local Variables:
%%% mode: latex
%%% TeX-master: "../index"
%%% End:

We can ensure that El-Gamal will sign on $\alpha(x)$ due to the
following:

As $\beta$ is defined as $\beta = \alpha^{a}$ and from the signature, it's given:
\begin{align*}
  \gamma &= \alpha^k \mod p \\
  \delta &= (x - a \gamma)k^{-1} \mod p - 1 \\
\end{align*}

And from the verification it's given:
\[ \beta^{\gamma}\gamma^{\delta} \equiv \alpha^x \mod p\]

\begin{align*}
  \alpha^x &\equiv \beta^{\gamma}\gamma^{\delta} \mod p\\
  &\equiv \alpha^{a^{\gamma}} \gamma^{(x-a\gamma)k^{-1}} \mod p\\
  &\equiv \alpha^{a\gamma} \alpha^{k(x-a\gamma)k^{-1}} \mod p\\
  &\equiv \alpha^{a\gamma} \alpha^{x-a\gamma} \mod p\\
  &\equiv \alpha^{a\gamma} \alpha^{x}\alpha^{-a\gamma} \mod p\\
  &\equiv \alpha^x \mod p
\end{align*}

Because of Fermat's little theorem (\ref{sec:fermats-little}) the
following is possible: if $\alpha\in \mathbb{Z}_{p'}^*$:
\begin{align*}
  \alpha^{p-1} &\equiv 1 \mod p\\
  \alpha^p &\equiv \alpha \mod p\\
  \text{then}&\\
  \alpha^x \mod p &\equiv \alpha^{x \mod p-1} \mod p
\end{align*}
