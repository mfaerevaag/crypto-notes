%%% Local Variables:
%%% mode: latex
%%% TeX-master: "../index"
%%% End:

\begin{itemize}
\item If attacker can factor $n$ then he can find $d$ same way that
    the system computes it
\item Attacker finds $\phi (n)$, then he can compute $d$ and find factors of $n$ and thereby break the system.
  \begin{itemize}
  \item If $n$ is small attacker can setup an equation with two unknowns
    that uses $\phi (n)=(p-1)(q-1)$ since he knows $n=pq$
    \begin{align*}
      \phi(n) &= n - p - q + 1 \\
      p &= n - \phi(n) - q + 1 \quad \text{since } p = \frac{n}{q} \\
      \frac{n}{q} &= n - \phi(n) - q + 1 \\
      0 &= q^2 + q(\phi(n) - n - 1) + n
    \end{align*}
  \end{itemize}
\item Attacker finds the decryption exponent $d$, factors $n$ and breaks the system
\end{itemize}