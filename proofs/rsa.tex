%%% Local Variables: 
%%% mode: latex
%%% TeX-master: "../index"
%%% End: 

\textbf{Public:} $n = pq$ for primes $p, q$ and $e \in \mathbb{Z}_{\phi(n)}^*$

\textbf{Private:} $(p, q, d)$ where $d \in \mathbb{Z}_{\phi(n)}^*$, such that
\[ ed \equiv 1 \mod \phi(n) \]

Define $\phi(n) = (p-1)(q-1)$. We have that
\begin{align*}
  ed \equiv 1 \mod \phi(n)\\
  \Rightarrow ed = 1 + k(p - 1)(q - 1)
\end{align*}

We need to prove that $(m^e)^d \equiv m \mod$ since that is enc and dec using RSA.


\begin{align*}
(m^e)^d \equiv m \mod n
\Rightarrow m^{ed} \equiv m \mod n
\end{align*}

Chinese remainder says that it is enough to show
\[ m^{ed} \equiv m \mod p \textbf{ and } m^{ed} \equiv m \mod q \]

Showing for $p$. There are two cases: 1) $m \equiv 0 \mod p$ and 2) $m \not\equiv 0 \mod p$.

\textbf{Case 1}
\begin{align*}
m &\equiv 0 \mod p\\
\Rightarrow m^x &\equiv m \mod p \quad \text{where } x \in \mathbb{Z}
\end{align*}
and since $ed \in \mathbb{Z}$ case 1 holds.

\textbf{Case 2}
\begin{align*}
m^{ed} &\equiv m \mod p\\
m^{1 + k(p-1)(q-1)} &\equiv m \mod p\\
m \cdot m^{k(p-1)(q-1)} &\equiv m \mod p\\
m \cdot (m^{(p-1)})^{k(q-1)} &\equiv m \mod p
\end{align*}

From \textbf{Fermats little theorem} we know that $x^{p-1} \equiv x \mod p$ for prime $p$. Thus
\begin{align*}
m \cdot 1^{k(q-1)} &\equiv m \mod p\\
m \cdot 1 &\equiv m \mod p\\
m &\equiv m \mod p
\end{align*}
