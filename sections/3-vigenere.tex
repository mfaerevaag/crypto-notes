%%% Local Variables:
%%% mode: latex
%%% TeX-master: "../index"
%%% End:

% Recommended:
% Explain Vigenere
% Must include cryptoanalysis
% Remember formulas
% Index of coincidence

\subsection*{Agenda}
\begin{enumerate}
\item Polyalphabetic substitution cipher
\item Vigenere
\item Cryptoanalysis
\end{enumerate}

\subsection{Polyalphabetic substitution cipher}

Polyalphabetic substitution cipher is an extension of mono alphabetic
substitution ciphers, where one character is encryptet at a time. Most
common is Caesar system.

Polyalphabetic systems encrypts more than one character, using
multiple alphabets for security. Thus the same plaintext character can
encrypt to different ciphertext characters.

\subsection{Vigenere crypto system}

Assume $\mathcal{P} = \mathcal{C} = \mathcal{K} =
\mathbb{Z}_{26}^j$. Encryption and decryption will be:
\begin{align*}
  e_{k}(m) \; &= \; (m_0+k_0 +m_1+k_1 + \ldots
  +m_{j-1}+k_{j-1} )  \mod 26\\
  d_{k}(c) \; &= \; (c_0-k_0 +c_1-k_1 + \ldots +c_{j-1}-k_{j-1} ) \mod 26
\end{align*}
where it's given that $m=(m_0,m_1 \ldots,m_{j-1})$, $=(k_0,k_1
\ldots,k_{j-1})$ and $m=(c_0,c_1 \ldots,c_{j-1})$. If the plaintext is
longer than $j$, the plaintext is split into $j$ blocks. If last block
is less than $j$ characters, part of the key $k$ are used for
encryption.

The values of the secret key is $26^j$. Cryptanalysis is still easy,
if the value of $j$ isn't too large.
\subsection{Cryptanalysis}

\begin{enumerate}
\item Find probable key period $l$ using IoC
\item Divide $m$ into $l$ blocks, find probable shift in each using
  letter distribution
\item Assemble the plaintext from the blocks
\end{enumerate}

\subsubsection{Index of Coincidence (IoC)}
To break the Vigenere cryptosystem we must look at the probability
distribution of the English language. This is the likehood of picking
the same letter twice in a row at random.

\[ I_c(\vec{x}) =
\frac{\sum^{m-1}_{i=0}\binom{f_i}{2}}{\binom{n}{2}}=
\frac{\sum^{m-1}_{i=0}f_i(f_i - 1)}{n(n-1)}\]

For a random text the value $I_c$ is

\[ I_c(\vec{x}) =
\frac{\sum^{25}_{i=0}f_i(f_i - 1)}{n(n-1)} \approx
\sum^{25}_{i0}(\frac{1}{26})^2= 1/26 \simeq 0.038\]

And for a natural language english $I_c = 0.065$

Split the text into $l$ blocks and compute $I_c$ for each resulting
vector. The $l$ that produces $I_c$ values that are closest to natural
english is likely to be the actual key period.

\subsubsection{Letter distributions}

Given vectors $x_i \ldots x_{l-1}$ we now compute the likely shift
length of each. That is, for each $x_i$ we do the following.

For each shift $1 \ldots 26$ compute the letter distribution, which is
the frequency of each letter $A \ldots Z$ in $x_i$. We then compute
the distance to the distribution of natural English. One way is ``sum
of squared differences''. Given $\overrightarrow{a}$ and
$\overrightarrow{b}$ the distance is

\[\Sigma_{n=o}^i (a_n - b_n)^2 \]

We choose the shift that has the smallest distance to natural english.

\subsubsection{Reassembly}
It is now trivial to reconstruct the full plaintext by appending
letters from the resulting shifted vectors alternately.