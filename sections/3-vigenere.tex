%%% Local Variables:
%%% mode: latex
%%% TeX-master: "../index"
%%% End:

\subsection*{Agenda}
% \begin{itemize}
% \item Explain Vigenere
% \item Must include cryptoanalysis
%   \begin{itemize}
%   \item Remember formulas
%   \item Index of coincidence
%   \end{itemize}
% \end{itemize} 

\begin{enumerate}
\item Poly alphabetic substitution cipher.
\item Vigenere crypto system
\item Cryptanalysis
\end{enumerate}

\subsection{Poly alphabetic substitution cipher}

Poly alphabetic substitution cipher is an extension of mono alphabetic
substitution ciphers, where one character is encryptet at a time. Most common is Caesar system.

Poly alphabetic systems encrypts more than one character, using
multiple alphabets for security.

\subsection{Vigenere crypto system}

Assume $\mathcal{P} = \mathcal{C} = \mathcal{K} =
\mathbb{Z}_{26}^j$. Encryption and decryption will be:
\begin{align*}
  e_{k}(m) \; &= \; (m_0+k_0 +m_1+k_1 + \ldots
  +m_{j-1}+k_{j-1} )  \mod 26\\
  d_{k}(c) \; &= \; (c_0-k_0 +c_1-k_1 + \ldots +c_{j-1}-k_{j-1} ) \mod 26
\end{align*}
where it's given that $m=(m_0,m_1 \ldots,m_{j-1})$, $=(k_0,k_1
\ldots,k_{j-1})$ and $m=(c_0,c_1 \ldots,c_{j-1})$. If the plaintext is
longer than $j$, the plaintext is split into $j$ blocks. If last block
is less than $j$ characters, part of the key $k$ are used for
encryption.

The values of the secret key is $26^j$. Cryptanalysis is still easy,
if the value of $j$ isn't too large.
\subsection{Cryptanalysis}
To break the Vigenere cryptosystem we must look at the probability
distribution of the English language. 
\subsubsection{IOC - Index of coincidence}

\[ I_c(\vec{x}) =
\frac{\sum^{m-1}_{i=0}\binom{f_i}{2}}{\binom{n}{2}}=
\frac{\sum^{m-1}_{i=0}f_i(f_i - 1)}{n(n-1)}\] 
n
For the English language:

\[ I_c(\vec{x}) =
\frac{\sum^{25}_{i=0}f_i(f_i - 1)}{n(n-1)} \approx
\sum^{25}_{i0}(1/26)^2= 1/26 \simeq 0.038\] 
