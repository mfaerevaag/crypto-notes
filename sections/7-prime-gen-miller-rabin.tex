%%% Local Variables:
%%% mode: latex
%%% TeX-master: "../index"
%%% End:

% Recommended:
% Prime generation
% Trial division
% Miller-Rabin
% - Not guaranteed correct, but chance of fault can be made small.

\subsection*{Agenda}
\begin{enumerate}
\item Prime generation
\item Trial division
\item Miller-Rabin
\end{enumerate}

\subsection{Prime Generation}
\subsection{Trial division}
Trial division is quite simple
\begin{itemize}
\item input an integer m
\item all primes $r\le \sqrt{m}$
\item if $r$ devides $m$ then m isn't prime
\item if all primes under are coprime then $m$ is prime
\end{itemize}

\subsection{Miller-Rabin}
Miller rabin is another method of determining if a number is prime or not. It does not suffer from the weakness of rating Carmichael numbers as primes. When checking for primes we only check odd numbers that are greater than 2.\\
Miller Rabin needs the following to function
\begin{itemize}
\item an odd m to check if prime
\item $s$ and $t$ which are found by subtracting 1 from m and dividing with two until result is odd. $s$ is the times we divide by 2 and $t$ is the odd result.
\item Now we have $m$, $s$, $t$ we now need to compute $b^t\mod m$
\item $b$ is a random int in the range $0<b<m$
\item now we have two cases
\begin{align*}
&\mbox{case 1}\\
&b^t \equiv 1 \mod m\\
&\mbox{case 2}\\
&b^{2^{j}t} \equiv -1 \mod m \mbox{ for at least one $j=0,...,s-1$}
\end{align*}
\item Even if $m$ succeeds in the rules above we cannot be completely sure if it really is a prime. If Miller Rabin says that $m$ is composit it is. But it can be shown that Miller Rabin has at most 1/4 probability of wrongly assigning a composit a prime. This can be countered with doing Miller Rabin $l$ iterations. This lowers the chance of the algorithm begin wrong by $(\frac{1}{4})^l$

\end{itemize}