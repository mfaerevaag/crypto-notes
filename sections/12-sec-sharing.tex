%%% Local Variables:
%%% mode: latex
%%% TeX-master: "../index"
%%% End:

% Recommended:
% $(t, w)$
% - Why can $t$-people reconstruct the key?
% - Why cannot less than $t$-people reconstruct the key?

\subsection*{Agenda}
\begin{enumerate}
\item Purpose
\item Shamir's threshold scheme
\item Key re-construction
\end{enumerate}

\subsection{Purpose}
What if you could give  only shares of the secret key to your
friends, but only need a subset of the shares to reconstruct the
secret key? Now none of your friends can alone decrypt you data, and
it doesn't matter if a friend or two loses their share. This is
\emph{secret sharing}.

\subsection{Shamir's threshold scheme}
\subsubsection*{Setup}
\begin{itemize}
\item $(t, w)$, where $t$ is the threshold and $w$ is the number of shareholders
\item Participants: Keyholder $H$, shareholders $S_1, \ldots, S_w$
\item Secret key: $k$
\item H chooses:
  \begin{itemize}
  \item a prime $p$, such that $p \ge w + 1$
  \item a random polynomial
    \[
    f(x) = k + \sum\limits_{j=1}^{t-1} c_jx^j \mod p
    \]
    of degree $t-1$, where $c_j$s are chosen at random
  \end{itemize}
\item $H$ computes $(x_i, f(x_i))$ for $i  = 1, \ldots, t$
\end{itemize}

\subsubsection*{Re-construction}
\begin{figure}[H]
  \begin{equation}\label{eq:lagrange}
    f(x) = \sum\limits_{i=1}^t y_1 \prod\limits_{j=1, j\not=i}^t \frac{x-x_j}{x_i-x_j}
  \end{equation}
  \label{asdf}
  \caption{Lagrange Interpolation Formula}
\end{figure}

\begin{itemize}
\item Input: $t$ shares $(x_i, y_i)$ for $i = 1, \ldots, t$
\item $H$ re-constructs the secret key $k$ by using Lagrange Interpolation Formula (\ref{eq:lagrange})
  \[ k = f(0) = \sum\limits_{i=1}^t y_1 \prod\limits_{j=1, j\not=i}^t \frac{x_j}{x_j-x_i} \]
\end{itemize}

\textbf{$t$ shareholders} \\
There is only one polynomial $f(x)$ over $\mathbb{Z}_p$ of degree at
most $t-1$, such that $f(x_i) = y_i$ for $i = 1,\ \dots,\
t$. Therefore, $t$ shareholders with the shares $\{(x_1,y_1),\
(x_2,y_2),\ \ldots,\ (x_t,y_t)\}$, can re-construct the polynomial
$f(x)$ using the Lagrange Interpolation Formula~(\ref{eq:lagrange}) and
compute the secret key for $f(0)$.

Further, if we describe the shares as follows
\begin{equation}
  \begin{pmatrix}
    1      & x_1    & x_1^2    & \cdots & x_1^{t-1} \\
    1      & x_2    & x_2^2    & \cdots & x_2^{t-1} \\
    \cdots & \cdots & \cdots  & \cdots & \cdots \\
    1      & x_{t-1} & x_{t-1}^2 & \cdots & x_{t-1}^{t-1} \\
    1      & x_t    & x_t^2    & \cdots & x_t^{t-1}
  \end{pmatrix}
  \begin{pmatrix}
    c_0 \\
    c_1 \\
    c_2 \\
    \cdots \\
    c_{t-1}
  \end{pmatrix} =
  \begin{pmatrix}
    y_0 \\
    y_1 \\
    y_2 \\
    \cdots \\
    y_t
  \end{pmatrix} \mod p
\end{equation}

Since $p$ is assumed to be a prime, and since $x_i \not= x_j$ for $1 \le i
< j \le t$, Theorem 2.4.3 tells us that the determinant of the matrix
is nonzero. As we also know that any nonzero element modulo $p$ (a
prime) has a multiplicative inverse, so the matrix above can be
inverted. Thus, re-constructing the polynomial $f(x)$.
\\

\textbf{$t-1$ shareholders} \\
If $t-1$ shareholders pool their shares, denoted $\{(x_1,y_1),\
(x_2,y_2),\ \ldots,\ (x_{t-1},y_{t-1})\}$, and determine that $f(0) = k'$.

\begin{equation}
  \begin{pmatrix}
    1      & x_1    & x_1^2    & \cdots & x_1^{t-1} \\
    1      & x_2    & x_2^2    & \cdots & x_2^{t-1} \\
    \cdots & \cdots & \cdots  & \cdots & \cdots \\
    1      & x_{t-1} & x_{t-1}^2 & \cdots & x_{t-1}^{t-1} \\
    1      & 0      & 0       & \cdots & 0
  \end{pmatrix}
  \begin{pmatrix}
    c_0 \\
    c_1 \\
    c_2 \\
    \cdots \\
    c_{t-1}
  \end{pmatrix} =
  \begin{pmatrix}
    y_0 \\
    y_1 \\
    y_2 \\
    \cdots \\
    k'
  \end{pmatrix} \mod p
\end{equation}

As for the same argument as above, for any value of $k'$ we can
re-construct an unique polynomial that could have generated their
shares, for which $f(0) = k'$. Hence, the shareholders have no
information about the value of the secret key.
