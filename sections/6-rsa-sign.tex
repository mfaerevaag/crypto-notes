%%% Local Variables:
%%% mode: latex
%%% TeX-master: "../index"
%%% End:

% Recommended:
% Similar to RSA encryption
% Why use hash function?
% - Harder to find preimage
% - Faster
% - Smaller signatures

\subsection*{Agenda}
\begin{enumerate}
\item Setup
\item Hash functions
\item Proof
\end{enumerate}

\subsection{Setup}
$H$ is a publicly known hash function, must be secure.

\textbf{Public key:} $(n, e)$ where $n = pq$ for primes $p, q$ and $e \in \mathbb{Z}_{\phi(n)}^*$

\textbf{Private key:} $(p, q, d)$ where $d \in \mathbb{Z}_{\phi(n)}^*$, such that
\[ ed \equiv 1 \mod \phi(n) \]

We use $\phi(n)$ because it is hard to compute given only $n$, but
easy with known $p, q$ since $\phi(n) = (p-1)(q-1)$, thus $d$ cannot
be easily computed given just $n$ (must prime factor $n$).

\textbf{Signature:} $m$ is the message and $x = H(m) \in \mathbb{Z}_n$. Signature is
\[ s = x^d \mod n \]

\textbf{Verification:} $m$ is the message, signature is $s \in \mathbb{Z}_n$. Accept if
\[ x \equiv s^e \mod n \]

\subsection{Hash functions}
$H$ is used to prevent the trivial forgey where Eve can choose any $s$
such that $s = e_{pub}(m)$. She cannot do this since she would need a
preimage for $H(m)$ as $s$ otherwise does not sign $H(m)$ but $m$.

\subsection{Proof}
%%% Local Variables:
%%% mode: latex
%%% TeX-master: "../index"
%%% End:

\textbf{Public:} $n = pq$ for primes $p, q$ and encryption exponent $e
\in \mathbb{Z}_{\phi(n)}^*$

\textbf{Private:} $(p, q, d)$ where $d \in \mathbb{Z}_{\phi(n)}^*$, such that
\[ ed \equiv 1 \mod \phi(n) \]

Define $\phi(n) = (p-1)(q-1)$. We have that
\begin{align*}
  & ed \equiv 1 \mod \phi(n)\\
  \Rightarrow\quad& ed = 1 + k(p - 1)(q - 1) \quad \text{where } k \in \mathbb{Z}
\end{align*}

We need to prove that $(m^e)^d \equiv m \mod$ since that is enc and dec using RSA.
\begin{align*}
(m^e)^d &\equiv m \mod n \\
\Rightarrow\quad m^{ed} &\equiv m \mod n
\end{align*}

Chinese remainder says that it is enough to show
\[ m^{ed} \equiv m \mod p \textbf{ and } m^{ed} \equiv m \mod q \]

Showing for $p$. There are two cases: 1) $m \equiv 0 \mod p$ and 2) $m \not\equiv 0 \mod p$.

\textbf{Case 1}
\begin{align*}
m &\equiv 0 \mod p\\
\Rightarrow\quad m^x &\equiv m \mod p \quad \text{where } x \in \mathbb{Z}
\end{align*}
and since $ed \in \mathbb{Z}$ case 1 holds.

\textbf{Case 2}
\begin{align*}
m^{ed} &\equiv m \mod p\\
m^{1 + k(p-1)(q-1)} &\equiv m \mod p\\
m \cdot m^{k(p-1)(q-1)} &\equiv m \mod p\\
m \cdot (m^{(p-1)})^{k(q-1)} &\equiv m \mod p
\end{align*}

From \textbf{Fermats little theorem} (\ref{sec:fermats-little}) we
know that $x^{p-1} \equiv x \mod p$ for prime $p$. Thus
\begin{align*}
m \cdot 1^{k(q-1)} &\equiv m \mod p\\
m \cdot 1 &\equiv m \mod p\\
m &\equiv m \mod p
\end{align*}

Therefore, case 2 also holds and we have proved for prime $p$.
