%%% Local Variables:
%%% mode: latex
%%% TeX-master: "../index"
%%% End:

\subsection{Euclid's Algorithm}
\label{sec:euclids-algorithm}
Finds $\gcd(a, b)$
\begin{align*}
  a = 90,&\quad b = 34 \\
  r_0 &= 90\\
  r_1 &= 34\\
  r_2 &= 90 - 2*34 = 22\\
  r_3 &= 34 - 1*22 = 12\\
  r_4 &= 22 - 1*12 = 10\\
  r_5 &= 12 - 1*10 = 2
\end{align*}
End because 2 divides 10, thus $\gcd(90, 34) = 2$

\subsection{Euclid's Extended Algorithm}
\label{sec:euclids-extended}
\begin{align*}
  s_j = \begin{cases}
    1 &\text{if}\ j = 0\\
    0 &\text{if}\ j = 1\\
    s_{j-2} - q_{j-1}s_{j-1} &\mbox{if} j \ge 2
  \end{cases}\\ \\
  t_j = \begin{cases}
    1 &\text{if}\ j = 0\\
    0 &\text{if}\ j = 1\\
    t_{j-2} - q_{j-1}t_{j-1} &\text{if}\ j \ge 2
  \end{cases}
\end{align*}

\begin{table}[H]
  \centering
  \begin{tabular}{lllll}
    $i$ & $r_i$ & $q_i$ & $s_i$ & $t_i$ \\ \hline
    0   & 90    &       & 1     & 0     \\
    1   & 34    & 2     & 0     & 1     \\
    2   & 22    & 1     & 1     & -2    \\
    3   & 12    & 1     & -1    & 3     \\
    4   & 10    & 1     & 2     & -5    \\
    5   & 2     & 1     & -3    & 8     \\
  \end{tabular}
  \caption{Example run of Euclid's Extended Algorithm}
\end{table}
The output of Euclid's extended algorithm can be written as:
\[ 1 = r \cdot s + t \cdot p\]
Where t will be the multiplicative inverse of $p^{-1} \mod s$.

\subsection{Chinese Remainder Theorem}
\label{sec:crm}
If $x^ay^b \equiv z \mod p$, then $(x^a \mod p)(y^b \mod p) \equiv z \mod p$

Also $x^ay^b \mod p = (x^a \mod p)(y^b \mod p) \mod p$

\subsection{Modular Multiplicative Inverse}
\label{sec:modular-mult-inverse}
Modular multiplicative inverse of $a \mod m$ is $x$ in
\[ ax \equiv 1 \mod m \]
where $x = a^{-1} \mod m$.

$a$ must be ``coprime'' to $m$ ($\gcd(a, m) = 1$).

\textbf{Ex} $3^{-1} \mod m = 4$, because $3 \cdot 4 = 12 \equiv 1 \mod 4$

Can be found using Euclid's Extended Algorithm~(\ref{sec:euclids-extended}).
\[ 1 = r \cdot s + t \cdot p\]
where t is the multiplicative inverse.
\subsection{Euler's $\phi$-function}
\label{sec:eulers-phi}
Defines the number of $a \in \mathbb{Z}_p$ for which $\gcd(a, p) = 1$,
e.i. number of coprimes to $p$ in $\mathbb{Z}_p$.

\subsection{Fermat's Little Theorem}
\label{sec:fermats-little}
For all $b \in \mathbb{Z}_p^*$ it holds that $b^{p-1} \equiv 1 \mod
p$, where $p$ is a prime.

\subsubsection*{Proof}
Given the set
\[ \mathbb{Z}_p^* = \{1,2,3,\ldots,p-1\} \]

we can multiply some element $b \in \mathbb{Z}_p^*$ onto all elements,
and get the set
\[ \{b,2b,\ldots,b(p-1)\} \mod p \]
since we know that none of the elements is congruent to zero modulo
$p$ from Theorem 2.4.3 we know $ab \equiv 0 \mod p$ if $a \lor b \equiv 0 \mod
p$.

From that it is also known that both sets are equal. Since
multiplication is commutative, we get:
\begin{align*}
  1 \cdot 2 \cdot \ldots \cdot (p - 1) \mod p &= b \cdot 2b \cdot \ldots
  \cdot b(p-1) \mod p\\
  &= b^{p-1}(1\cdot 2 \cdot \ldots \cdot (p - 1)) \mod p
\end{align*}

From the before mentioned theorem, it is also known that the left-hand
side has a multiplicative inverse due to $1 \cdot 2 \cdot \ldots \cdot
(p - 1) \not \equiv 0 \mod p$. If that is multiplied on each side we
get Fermat's Little Theorem:
\[b^{p-1} \equiv 1 \mod p \]

\subsection{Theorem 2.4.3}
Let $p$ be a prime and let $a,b \in \mathbb{Z}$ Then it holds that
\[ ab \equiv 0 \mod p \Rightarrow a \equiv 0 \mod p \; \lor \; b
\equiv 0 \mod p\]

\subsubsection*{Proof}
Assume $ab \equiv 0 \mod p$.

If $a \equiv 0 \mod p$ we are done. So we need to show nif $a \not
\equiv 0 \mod p$ then $b \equiv 0 \mod p$.

Since $a \not \equiv 0$ for $a \in \mathbb{Z}_p^*$, it has a
multiplicative inverse modulo $p$
\[ ac \equiv 1 \mod p \]

Thus we can see
\[ b = 1 \cdot b = (ca)b=c(ab) = c \cdot 0 = 0 \]