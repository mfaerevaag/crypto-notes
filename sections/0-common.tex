%%% Local Variables: 
%%% mode: latex
%%% TeX-master: "../index"
%%% End: 

\subsection{Euclids algorithm}
Finds $\gcd(a, b)$

\begin{align*}
  a = 90,& b = 34 \\
  r_0 &= 90\\
  r_1 &= 34\\
  r_2 &= 90 - 2*34 = 22\\
  r_3 &= 34 - 1*22 = 12\\
  r_4 &= 22 - 1*12 = 10\\
  r_5 &= 12 - 1*10 = 2
\end{align*}
End because 2 divides 10, thus $\gcd(90, 34) = 2$

\subsection{Euclids extended algorithm}
\begin{align*}
  s_j = \begin{cases}
    1 &\mbox{if} j = 0\\
    0 &\mbox{if} j = 1\\
    s_{j-2} - q_{j-1}s_{j-1} &\mbox{if} j \ge 2
  \end{cases}\\ \\
  t_j = \begin{cases}
    1 &\mbox{if} j = 0\\
    0 &\mbox{if} j = 1\\
    t_{j-2} - q_{j-1}t_{j-1} &\mbox{if} j \ge 2
  \end{cases}
\end{align*}

\begin{table}[H]
  \centering
  \begin{tabular}{lllll}
    $i$ & $r_i$ & $q_i$ & $s_i$ & $t_i$ \\ \hline
    0   & 90    &       & 1     & 0     \\
    1   & 34    & 2     & 0     & 1     \\
    2   & 22    & 1     & 1     & -2    \\
    3   & 12    & 1     & -1    & 3     \\
    4   & 10    & 1     & 2     & -5    \\
    5   & 2     & 1     & -3    & 8     \\
  \end{tabular}
  \caption{Example run of euclids extended}
\end{table}

\subsection{Chinese remainder}
If $x^ay^b \equiv z \mod p$, then $(x^a \mod p)(y^b \mod p) \equiv z \mod p$

Also $x^ay^b \mod p = (x^a \mod p)(y^b \mod p) \mod p$

\subsection{Modular multiplicate inverse}
Modular multiplicative inverse of $a \mod m$ ($a^{-1} \mod m$), is $x$ where
\[
ax \equiv 1 \mod m
\]

$a$ must be ``coprime'' to $m$ ($\gcd(a, m) = 1$).

\textbf{Ex} $3^{-1} \mod m = 4$, because $3 \cdot 4 = 12 \equiv 1 \mod 4$

Can be found using Euclids extended algorithm


\subsection{Eulers $\phi$}
Defines the number of $a \in \mathbb{Z}_p$ for which $\gcd(a, p) = 1$, ie number of coprimes to $p$ in $\mathbb{Z}_p$.

\subsection{Fermats little theorem}
For all $b \in \mathbb{Z}_p^*$ it holds that $b^{p-1} \equiv 1 \mod p$
